\documentclass[main.tex]{subfiles}
\begin{document}
\section{Introduction}

A time series is a sequence of data containing a temporal element, whether it be information about stocks, temperature measurements over a period, or the daily spread of a virus. In machine learning this is an interesting area of research as time series forecasting is a problem domain that consists of basically predicting the future based on historical data.\\
For a long time, based on time series data's sequential nature, the most widely used methods for time series forecasting have been recurrent methods such as recurrent neural networks and gated methods such as the long short term memory network. However, these types of methods still run into hurdles which make it hard for them to learn the long term dependencies that are key in predicting time series data. In an attempt to find new ways to tackle the challenge of "predicting the future", this project will try to examine the use of transformers and \textit{attention} for time series forecasting, comparing them to the more mainstream methods. In this project, the focus will be on predicting multiple time steps as this is where both model architectures will be sufficiently challenged and should give a better visualization of how the models perform differently. \\
\\
Each section will try to provide a theoretical background to each type of model, showing what makes them compatible for time series forecasting, and also showing the parts that make them different, and hopefully what differentiates their results from each other. From each type of model the results are presented and compared, and lastly their differences are evaluated and reflected on. To round of the project, a discussion on the method of approach is presented with the aim of giving the reader an idea of why different choices were made and why certain elements were left out.\\
This project is written with the mindset that the reader is knowledgeable on machine learning and data science, but is not necessarily an expert in the field of the ideas presented here, and on which this project hopefully should elaborate on.

\end{document}